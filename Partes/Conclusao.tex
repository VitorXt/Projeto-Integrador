\chapter{\uppercase{Considerações finais}}
\label{conclusao}

A metodologia utilizada em sala para o desenvolvimento das atividades relacionadas ao trabalho de Projeto Integrador foi o SCRUM. \cite{scrum}\\

Em um mundo onde as pessoas estão cada vez mais envolvidas com a tecnologia, é de suma importância que os usuários estejam cientes de que para uma boa navegação é necessário ter conhecimento sobre os riscos que a internet oferece, como os dados ficam armazenados na web e quais são os cuidados que cada um deve ter ao acessar esses meios.  \cite{cookies}

Este projeto foi desenvolvido pensando em levar a informação para todos os tipos de usuário, sejam eles mais experientes ou menos. A criação do Bot Security teve como objetivo dois princípios: abordar sobre os principais tipos de golpes que estão sendo aplicados atualmente por meio da internet e oferecer algumas dicas para que o usuário tenha uma experiência mais segura na Web.\cite{segint} 

A entrega final do trabalho foi disponibilizada em uma rede social, o Instagram. Foi criada uma conta para divulgar o Bot e também abordar alguns temas estudados no 2º Período do Projeto Integrador para complementar o perfil e consequentemente levar mais conhecimento aos seguidores.

Apesar do objetivo final ter sido alcançado, houve alguns impasses durante a execução do projeto que fez com que o resultado final não saísse completamente da forma como planejado. O Bot teve que ser recriado e configurado 3 vezes devido as primeiras plataformas usadas para sua criação não serem gratuitas, o que gerou alguns atrasos e muito retrabalho. Em consequência deste atraso, alguns detalhes finais tais como configurações no Bot e plano de testes não puderam receber tanta atenção. Diante destes cenários, é concluído que para próximos trabalhos é preciso mais cautela e pesquisa antes de escolher plataformas que serão usadas no projeto. 

Em consideração as observações e pesquisas realizadas, foi possível detectar que a “Segurança na Internet” engloba aspectos bem mais amplos, pois envolve uma variedade de informações, dados pessoais e proteção à propriedade, e que se faz necessário aplicar conhecimento que visem ampliar a visão do usuário para que o mesmo busque mais entendimento, tornando-se consciente de suas ações na Web e contribuindo para um ambiente on-line pacífico.


 \section{Trabalhos Futuros}
  
Futuramente este trabalho pode ser complementado através de um aprimoramento nas configurações do BOT, como implementação de Inteligência Artificial, pois, atualmente o BOT está limitado a um menu de opções onde o usuário tem apenas a possibilidade de escolher uma das opções apresentadas. E com a Inteligência Artificial o usuário poderá ter uma experiência melhor e semelhante a de estar conversando com um atendente. 












